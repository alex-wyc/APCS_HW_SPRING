\documentclass [12 pt, twoside] {article}
\usepackage[margin=1in]{geometry}
\usepackage[utf8]{inputenc}
\usepackage{listings}
\usepackage{color}
\usepackage{setspace}
\usepackage{indentfirst}

\definecolor{codegreen}{rgb}{0,0.6,0}
\definecolor{codegray}{rgb}{0.5,0.5,0.5}
\definecolor{codeblue}{rgb}{0,0,0.6}
\definecolor{backcolor}{rgb}{0.95,0.95,0.95}

\lstdefinestyle{mystyle}{
	backgroundcolor = \color{backcolor},
	commentstyle = \color{codeblue},
	keywordstyle = \color{codegreen},
	numberstyle = \color{codegray},
	stringstyle = \color{magenta},
	basicstyle = \footnotesize,
	breakatwhitespace = false,
	breaklines = true,
	captionpos = b,
	keepspaces = true,
	numbers = left,
	numbersep = 5pt,
	showspaces = false,
	showstringspaces = false,
	showtabs = false,
	tabsize = 4
}

\lstset{style = mystyle}

\begin{document}

\title{APCS Notes}
\author{Yicheng Wang}
\date{2014-2015}

\maketitle
\newpage
\setcounter{tocdepth}{3}
\tableofcontents
\newpage

\section{2015-02-05}
\subsection{Do Now}
Figure out what the following code does.

\begin{lstlisting} [language=Java]
public void printme(int n) {
	if (N > 0) {
		printme(n - 1);
		System.out.println(n);
	}
}
\end{lstlisting}

It should print out an increasing sequence of numbers
from 1-N.

\subsection{Stack and ROP}
The stack on top is the current function, and each layer beneath that
is the function that called the current function.

\subsection{Recursion}



Simple recursive problem: FACTORIAL


Hallmarks of a recursive solutioin:
\begin{itemize}
	\item Base Case: thing that stops the program, simple case you know the answer of. In the case of factorials, factorial(0) = 1
	\item Reduction Case: You need to alternate the variable in some sort of way, for example, we should do n * factorial(n - 1)
	\item Recursion: function A need to eventually call A
\end{itemize}


Final code:
\begin{lstlisting}[language=Java]
public int factorial(n) {
	if (n == 0) {
		return 1; // Base Case
	}
	else {
		return n * factorial(n - 1); // Reduction Step
	}
}
\end{lstlisting}

\section{2015-02-06}

\subsection{Traditional Recursion Examples}
\subsubsection{Fibonacci Numbers}
1, 1, 2, 3, 5, 8, 13 ...


Base Case: if n < 2, return 1


Reduction Step: fib(n) = fib(n - 1) + fib(n - 2)


Example Code:
\begin{lstlisting}[language=Java]
public int fib(int n) {
	if (n < 2) {
		return 1;
	}

	else {
		return fib(n - 1) + fib(n - 2);
	}
}
\end{lstlisting}

\subsubsection{List/String Manipulation}


Example, finding the length of a substring.


Base case: "" has length of 0


Reduction Step: 1 + "cdr" of the string, i.e. s.substring(1);


Example Code:
\begin{lstlisting}[language=Java]
public int lenStr(String s) {
	if (s.equals("")) {
		return 0;
	}
	else {
		return 1 + lenStr(s.substring(1));
	}
}
\end{lstlisting}

\section{2015-02-09}
\subsection{Getting out of a Maze}
\begin{enumerate}
	\item Maze vs. Laybrith: Laybrith may not have choices, mazes have choices.
	\item Strategies in solving the maze:
		\begin{itemize}
			\item Doesn't work due to loops
		\end{itemize}
	\item Greek Way of Solving Mazes:
		\begin{itemize}
			\item Invented by Odysseus, bring a thread, use process of elimination to loop through all possible intersection.
			\item Works really well.
		\end{itemize}
\end{enumerate}


Maze solving in java
\begin{enumerate}
	\item This is clearly a recursive solution.
	\item Using a recursive solution we can easily trace back the stack to find the previous intersection
	\item We represent our map as a char array of paths ('\#')
	\item Base case: location is a wall OR location of exit
	\item Reduction step: call solve at a different (x,y) location, if it's a dead end, it'll be peeled off the stack due to the "return"
	\item We will try to solve the maze systematically, x+/-1, y+/-1.
\end{enumerate}

\section{2015-02-11}
\subsection{Blind Search}

Trying all possibilities until we find our solutions. Also called exhaustive search, linear search,
recursive search, search with backtracking.


The maze algorithm we wrote is sometimes known as a depth first search. It's because
you say in one path for as long as possible. Advantage is if the solution is deep, this works
pretty well. However, if the solution is closer, breath first search goes n steps in all
possible directions.

\subsection{State Space Search}

Search algorithm to solve a problem, searching for a "state space."

Idea is if you have a problem, you can describe said problem as a "state."

State is a configuration of the world, such as turtle in netLogo.

The world is made up of many states, and one can transition from one state to the next.

\textbf{State Space Search:} The series of states one has to go through to get to the desirable final state. (like exit in the maze thing)

\subsection{Graph Theory}

Graphs are collections of edges and nodes. Nodes represent the states, edges marks the transition between one node to the next.


\section{2015-02-12}
\subsection{Space State Search}
Examples of space state search:
\begin{itemize}
	\item Maze path-finding
	\item 15 puzzle
	\item Cube
	\item Chess algorithm -- More complex
		\begin{itemize}
			\item two players!
			\item high branching factor
			\item harder base case
			\item uses mini-max searching: best for me and worst for you
		\end{itemize}
	\item Description of an entity
\end{itemize}

\subsection{Implicit Data Structure}

When we did the maze solver, we used an explicit data structure \to the 2D array.
However, the graph of the transition is also a data structure, it's implicit though,
just running in the background. Whenever we call solve, it creates a node on the
stack. However, if a call returns, it destroy that branch of the graph.

As the program runs, we can imagine it as a graph.

\subsection{PROJECT}
Do the diagonistic exam in the Barron's Review book, do as follows:
\begin{itemize}
	\item First do it under test condition
	\item Go back and tried to fix your problem
\end{itemize}

\textbf{Option 1. Knight Tour:}


Given a $n \times n$ board, find a path such that the knight visits
each square once without repetition. (Start with $5 x 5$);


\textbf{Option 2. N-Queen:}


A way to place N mutually non-attacking queens on a $N \times N$ chessboard.
\end{document}
