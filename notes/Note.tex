\documentclass [12 pt, twoside] {article}
\usepackage[margin=1in]{geometry}
\usepackage[utf8]{inputenc}
\usepackage{listings}
\usepackage{color}
\usepackage{setspace}
\usepackage{indentfirst}

\definecolor{codegreen}{rgb}{0,0.6,0}
\definecolor{codegray}{rgb}{0.5,0.5,0.5}
\definecolor{codeblue}{rgb}{0,0,0.6}
\definecolor{backcolor}{rgb}{0.95,0.95,0.95}

\lstdefinestyle{mystyle}{
	backgroundcolor = \color{backcolor},
	commentstyle = \color{codeblue},
	keywordstyle = \color{codegreen},
	numberstyle = \color{codegray},
	stringstyle = \color{magenta},
	basicstyle = \footnotesize\ttfamily,
	breakatwhitespace = false,
	breaklines = true,
	captionpos = b,
	keepspaces = true,
	numbers = left,
	numbersep = 5pt,
	showspaces = false,
	showstringspaces = false,
	showtabs = false,
	tabsize = 4
}

\lstset{style = mystyle}

\begin{document}

\title{APCS Notes}
\author{Yicheng Wang}
\date{2014-2015}

\maketitle
\newpage
\setcounter{tocdepth}{3}
\tableofcontents
\newpage

\section{2015-02-05}
\subsection{Do Now}
Figure out what the following code does.

\begin{lstlisting} [language=Java]
public void printme(int n) {
	if (N > 0) {
		printme(n - 1);
		System.out.println(n);
	}
}
\end{lstlisting}

It should print out an increasing sequence of numbers
from 1-N.

\subsection{Stack and ROP}
The stack on top is the current function, and each layer beneath that
is the function that called the current function.

\subsection{Recursion}



Simple recursive problem: FACTORIAL


Hallmarks of a recursive solutioin:
\begin{itemize}
	\item Base Case: thing that stops the program, simple case you know the answer of. In the case of factorials, factorial(0) = 1
	\item Reduction Case: You need to alternate the variable in some sort of way, for example, we should do n * factorial(n - 1)
	\item Recursion: function A need to eventually call A
\end{itemize}


Final code:
\begin{lstlisting}[language=Java]
public int factorial(n) {
	if (n == 0) {
		return 1; // Base Case
	}
	else {
		return n * factorial(n - 1); // Reduction Step
	}
}
\end{lstlisting}

\section{2015-02-06}

\subsection{Traditional Recursion Examples}
\subsubsection{Fibonacci Numbers}
1, 1, 2, 3, 5, 8, 13 ...


Base Case: if n < 2, return 1


Reduction Step: fib(n) = fib(n - 1) + fib(n - 2)


Example Code:
\begin{lstlisting}[language=Java]
public int fib(int n) {
	if (n < 2) {
		return 1;
	}

	else {
		return fib(n - 1) + fib(n - 2);
	}
}
\end{lstlisting}

\subsubsection{List/String Manipulation}


Example, finding the length of a substring.


Base case: "" has length of 0


Reduction Step: 1 + "cdr" of the string, i.e. s.substring(1);


Example Code:
\begin{lstlisting}[language=Java]
public int lenStr(String s) {
	if (s.equals("")) {
		return 0;
	}
	else {
		return 1 + lenStr(s.substring(1));
	}
}
\end{lstlisting}

\section{2015-02-09}
\subsection{Getting out of a Maze}
\begin{enumerate}
	\item Maze vs. Laybrith: Laybrith may not have choices, mazes have choices.
	\item Strategies in solving the maze:
		\begin{itemize}
			\item Doesn't work due to loops
		\end{itemize}
	\item Greek Way of Solving Mazes:
		\begin{itemize}
			\item Invented by Odysseus, bring a thread, use process of elimination to loop through all possible intersection.
			\item Works really well.
		\end{itemize}
\end{enumerate}


Maze solving in java
\begin{enumerate}
	\item This is clearly a recursive solution.
	\item Using a recursive solution we can easily trace back the stack to find the previous intersection
	\item We represent our map as a char array of paths ('\#')
	\item Base case: location is a wall OR location of exit
	\item Reduction step: call solve at a different (x,y) location, if it's a dead end, it'll be peeled off the stack due to the "return"
	\item We will try to solve the maze systematically, x+/-1, y+/-1.
\end{enumerate}

\section{2015-02-11}
\subsection{Blind Search}

Trying all possibilities until we find our solutions. Also called exhaustive search, linear search,
recursive search, search with backtracking.


The maze algorithm we wrote is sometimes known as a depth first search. It's because
you say in one path for as long as possible. Advantage is if the solution is deep, this works
pretty well. However, if the solution is closer, breath first search goes n steps in all
possible directions.

\subsection{State Space Search}

Search algorithm to solve a problem, searching for a "state space."

Idea is if you have a problem, you can describe said problem as a "state."

State is a configuration of the world, such as turtle in netLogo.

The world is made up of many states, and one can transition from one state to the next.

\textbf{State Space Search:} The series of states one has to go through to get to the desirable final state. (like exit in the maze thing)

\subsection{Graph Theory}

Graphs are collections of edges and nodes. Nodes represent the states, edges marks the transition between one node to the next.


\section{2015-02-12}
\subsection{Space State Search}
Examples of space state search:
\begin{itemize}
	\item Maze path-finding
	\item 15 puzzle
	\item Cube
	\item Chess algorithm -- More complex
		\begin{itemize}
			\item two players!
			\item high branching factor
			\item harder base case
			\item uses mini-max searching: best for me and worst for you
		\end{itemize}
	\item Description of an entity
\end{itemize}

\subsection{Implicit Data Structure}

When we did the maze solver, we used an explicit data structure $\to$ the 2D array.
However, the graph of the transition is also a data structure, it's implicit though,
just running in the background. Whenever we call solve, it creates a node on the
stack. However, if a call returns, it destroy that branch of the graph.

As the program runs, we can imagine it as a graph.

\subsection{PROJECT}
Do the diagonistic exam in the Barron's Review book, do as follows:
\begin{itemize}
	\item First do it under test condition
	\item Go back and tried to fix your problem
\end{itemize}

\textbf{Option 1. Knight Tour:}


Given a $N \times N$ board, find a path such that the knight visits
each square once without repetition. (Start with $5 \times 5$);


\textbf{Option 2. N-Queen:}


A way to place N mutually non-attacking queens on a $N \times N$ chessboard.

\section{2015-02-13}

Other problem, third variation, one can also do the 15 puzzle.

\subsection{System.out.printf}
When one is doing a knight's tour, the print output may be distorted if
one moves from single digit to double digit. We can leave a placeholder
in the format string, such as \%s, \%d, etc.


Example code:
\begin{lstlisting}[language=java]
...
System.out.printf("%s, helloWorld\n %s\n", "title", "more stuff!");
...

OUTPUT:

title, helloWorld
more stuff!
\end{lstlisting}

However, printf can take multiple types and it is possible to save spaces for
formatters. Like the following:

\begin{lstlisting}
...
System.out.printf("%3d\n%3d", 1, 123);
...

OUTPUT:

  1
123
\end{lstlisting}

\section{2015-02-25}

Iteninary:
\begin{itemize}
	\item Tomorrow - Knight's Tour
	\item Tuesday - USACO
\end{itemize}

Today, we're going back to sorting.

\section{Merge Sort}

So far, we've covered 3 algorithms: selection sort, insertion sort, and bubble sort.
They are all linear algorithms. Selection sort selects the ith index and puts the ith
smallest/largest elements there, whereas insertion sort inserts the ith smallest/largest
element so far into the ith index of the list.

HOWEVER, we're lazy and need to do this the 6 years old way. If we're sorting a deck of
unsorted cards, we'll split the deck in half and give it to more people, so on, and so
on. This continues until everyone has 1 card at hand. The process then goes backwards
and the people merges the sorted lists passed on to him/her. This is known as a merge sort,
it is a \textbf{divide and conquer algorithm.}

Sample code:
\begin{lstlisting}[language=java]
...
	public static int[] mergeSort(int[] data) {
        if (data.length == 1) {
            return data;
        }

        else {
            int[] A = Arrays.copyOfRange(data, 0, data.length / 2);
            int[] B = Arrays.copyOfRange(data, data.length / 2, data.length);

            int[] AS = mergeSort(A);
            int[] BS = mergeSort(B);
            return merge(AS, BS);
        }
    }

    public static int[] merge(int[] A, int[] B) {
        int[] result = new int[A.length + B.length];
        int position = 0;
        int APos = 0;
        int BPos = 0;
        while (APos < A.length && BPos < B.length) {
            if (A[APos] < B[BPos]) {
                result[position] = A[APos];
                APos++;
            }
            else {
                result[position] = B[BPos];
                BPos++;
            }
            position++;
        }

        for (int i = APos ; i < A.length ; i++) {
            result[position] = A[i];
            position++;
        }
        for (int i = BPos ; i < B.length ; i++) {
            result[position] = B[i];
            position++;
        }
        return result;
    }

...
\end{lstlisting}

\section{2015-03-04}
\subsection{On the Algorithm of the Merge Sort}
In insertion and selection sorts, the average operation has a complexity of
$n^2$. This is because the inner loop runs through the list ($n$ operations)
and the outer loop instructs us to run through the inner loop $n$ times, hence
the total operations is $n^2$.

For merge sort, each step we split the list in half and sort the smaller part
first. If we imagine this as a tree, we divide the list in half each time. So
the vertical step takes $\log_2{n}$ steps. Then at each step, we copy the array
once so there is also a $n$ component to the total complexity. Then, the sum
of each level's merge is also going to be $n$ because merging takes the same
amount of work as splitting. Therefore, the total complexity is:
$$O(n) = n\log{n}$$

Let's look at the different rate of growth of the two curves ($n^2$ and
$n\log{n}$). Take, 1,000,000 as $n$, $n^2 = 1.0 \times 10^{12}$, but $n\log{n} = 6
\times 10^{6}$

\subsection{Big O Notation}
A function $f(n)$ is said to be $O(g(n))$ if there exists some constant $k$
such that $kg(n) > f(n)$ over the long term.

Note that the Big O Notation is always and \textbf{Upper bound}, and is
extracted from the worst-case scenario of the function. However, it is a very
tight upper bound.

\section{2015-03-05}
We shall write an algorithm to find the $k^{th}$ smallest element.

A few ways to do this:

One, we can recursively delete the smallest data point, but this isn't very
efficient. In fact, this is a $O(n^2)$ algorithm.

The other, cheap (my) way is to first merge sort it and then just do a lookup
for the element. This has the complexity of only $O(n\log{n})$.

Note that for hard problems, it is very hard for complexity to go below
$O(n\log{n})$. Therefore, merge sorting something is practically free,
complexity-wise. Case in point: is it worthwhile to first sort an array and
then use binary search with the merge sort or just use the linear search.

However, there is a more efficient way of doing this.

We can use a binary-search-like algorithm. 

\section{2015-03-09}
\subsection{Efficiency of Quick Select}

Given that we are lucky, and if the pivot is good such that it bisects the
array, on the first operation we only go through $\frac{n}{2}$ operations after
going through the entire array. Then it just divides by 2 each time, so the
final operation number is:

$$n + \frac{1}{2}n + \frac{1}{4}n + ... = 2n$$

So the quickselect runs on linear time!

\section{2015-03-11}
\subsection{Linked List}

A linked list is comprised of elements/nodes and a way to get to the "start
node." A node contains some data and information on how to get to the next
node. There are different forms of linked list. We are going to make the scheme
list.

\subsection{Scheme-like list in Java}
\begin{lstlisting}[language=java]
public class Node {
    private String data;
    private Node next; // Self-referencing data structure

    public Node(string s) {
        data = s;
    }

    public void setData(String s) {
        data = s;
    }

    public String getData() {
        return data;
    }

    public void (Node n) {
        next = n;
    }

    public Node getNext() {
        return next;
    }

    public String toString() {
        return data;
    }
}

public class Driver {
    public static void main(String[] args) {
        Node n1 = new Node("Hello");
        Node n2 = new Node("World");

        System.out.println(n1.getNext().getData()); // prints out "World"

        n.getNext().setData("pickle") // changes n2 to pickle

        System.out.println(n1.getNext().getData()); // prints out "pickle"

        n2.setNext("abc") // Creates a new node and point the next of n2 to it.

        n1.getNext().getNext().setNext("blah");

        n0 = new Node("Start");
        n0.setNext(n1); // Prefixes n0 to the list
        // If we want n1 to refer to the start of the list, we do:
        n1 = n0;

        // Right now the list looks like this:
        // "Start" --> "Hello" --> "pickle" --> "blah"
        System.out.println(n1);
    }
}
\end{lstlisting}

\section{2015-03-11}

\subsection{Print out every element in a linked list}

To print everything from an array, we do:
\begin{lstlisting}[language=java]
int i = 0;
while (i < A.length) {
    System.out.println(A[i]);
    i++;
}
\end{lstlisting}

To do the same for a linked list:
\begin{lstlisting}[language=java]
Node tmp = n; // starting node
while (tmp != null) {
    System.out.println(tmp);
    tmp = tmp.getNext();
}

// With a for loop:
for (Node tmp = n ; tmp != null ; tmp = tmp.getNext()) {
    System.out.println(tmp);
}

// And recursively:
public static print(Node n) {
    if (n.getNext() == null) {
        System.out.println(n);
        return;
    }

    else {
        System.out.println(n);
        print(n.getNext())
    }
}
\end{lstlisting}

\section{2015-03-23}
\subsection{Data Structures}
\subsubsection{STACKS}
Stack based objects add on the top and take from the top. This is known as a
last-in-first-out (\textbf{LIFO}) data structure. Stacks are concepts, it's NOT a basic
data structure.

Traditionally, a stack has certain operations:
\begin{lstlisting}[language=java]
myStack<Integer> s = new myStack<Integer>();
s.push(10); // Push takes whatever your parameter is and pushes it onto 
            // the top of the stack

s.push(5); // Now 10 is no longer the top, now 5 is on top

i = s.pop(); // Take the item off of the top of the stack and return it.
System.out.println(i); // In this case i = 5 and 10 is back on top
\end{lstlisting}

\textbf{push} and \textbf{pop} are the two BASIC operations that all stack classes have to have,
but they usually gives out other methods such as \textbf{empty} (check if empty), \textbf{peek}
or \textbf{top} (returns but doesn't remove),  \textbf{size} (give how many
elements are in the stack), etc.

\subsection{Assignment}
Make a class called myStack that mimics a stack, make methods push, pop, empty
and peek. Use this as an underlying linked list.

\section{2015-03-25}
\subsection{Queue}
A queue, unlike a stack, is what is known as a first-in-first-out
(\textbf{FIFO}) data structure. The basic tradition terms for a queue are called
\textbf{enqueue} and \textbf{dequeue}. Enqueue is the equivalent of push and
dequeue is the equivalent to pop. There may also be \textbf{empty} and
\textbf{front}.

Front in a queue is more important than that of a stack, because one cannot
simply pop and then push for the queue. But front is still not considered a
basic command.

\subsection{Assignment}
Implement the queue with a single linked-list, and enqueue and dequeue should be
able to run at constant time.

\section{2015-03-31}
\subsection{Maze}
Review on the maze algorithm we did a few weeks ago:
\begin{enumerate}
    \item 2D array of chars
    \item Recursive Deepth First Search
\end{enumerate}


However, if we had infinite resources, it may be better to do a breadth first
search. This is a good strat when you suspect that the answer is closer by. This
is  known as breadth first search.


Example maze:
\begin{lstlisting}
       ---
       |B|
       ---
       |C|
     -----------
     |D|A|E|F|G|
     -----------
       |H|
       ---
       |I|
       ---------
       |J|K|M|I|
       ---------
\end{lstlisting}


We create a concept called the frontier, which is the "next square to be
explored."

If we start at square A in the example maze, the frontier would be consisted of
E, C, D, and H. However, for argument's sake let's say E was the first square to
be explored, after it is taken off of the frontier. the next square to be
explored would be one of C, D and H instead of F. So the frontier should be
stored as a FIFO system, i.e. a queue.

Basic code snippet:
\begin{lstlisting}[language=java]
    public void solve(Node position) {
        while (!frontier.empty()) {
            current = frontier.dequeue();
            if (current.equals(exit)) {
                System.exit(0);
            }
            ... ... ...
        }
    }
\end{lstlisting}

\section{2015-4-22}
\subsection{Boring Definition Day}
\subsubsection{Graph/Trees}
Trees are gateways to more powers

Tree is a specification of graph, a graph is a collection of nodes (or vertices) and edges.

A node holds stuff and an edge connects nodes.


\textbf{A tree} is an acyclic graph such that one node is defined as "the root."

A tree is either empty or has one or more nodes, each node can have 0 or more
children and one node is designated as the root.

\textbf{Leaf}: A node with 0 children.

\textbf{Siblings}: Nodes that share a parent (1 level up only)

\textbf{Path}: A sequence of edges/nodes that takes you from 1 place to another

\textbf{Height}: The length of the longest root-to-leaf path

\textbf{Full}: A tree is full means that every single node is either a leaf or
has the fullest amount of children

\textbf{Balance}: THe same number of children and node on both sides.

\subsubsection{Binary Tree}
\textbf{Binary Tree}: A tree in which each node is either a leaf or has 1 or 2
chidren.

\section{2015-4-23}
\subsection{Design of the Node}

The tree is based on a double-linked list, each node should have two private
Nodes left, and right and generic data.

\subsection{Binary Search Tree}

It is a binary tree such that given any Node w/ value v, all the node in the
left subtree has values < V and all nodes in the right subtree have values
greater than V.

\subsection{Sample Algorithms}
\begin{lstlisting}[language=java]
public Node Search (Node T, Integer i) {
    while (T != null) {
        int c = T.getData().compareTo(i);
        if (c == 0) {
            return T;
        }
        T = (c > 0) ? T.getRight() : T.getLeft();
    }
    return null;
}

public insert (Node T) {
    Node tmp1 = root;
    Node tmp2 = null;
    while (tmp1 != null) {
        int c = T.getData().compareTo(tmp1.getData());
        tmp2 = tmp1;
        tmp1 = (c > 0) ? tmp1.getRight() : tmp1.getLeft();
    }
    int c = T.getData().compareTo(tmp2.getData());
    if (c > 0) {
        tmp2.setRight(T);
    }
    else {
        tmp2.setLeft(T);
    }
}
\end{lstlisting}

\section{2015-4-29}
\subsection{Remove in a Tree}
\begin{lstlisting}
10
-5
|-2
||-4
|-8
-20
 -17
 |-15
 |-18
 -25
  -100
   -101
\end{lstlisting}
There are three situations when we try to delete a node, each one harder than
the next. However, we need to use the same piggy-back algorithm as used in
insertion.
\subsubsection{Leaf}
It is very easy to remove a leaf, we simply set the parent's left/right to null.

\subsubsection{1 Child}
This is a little bit more work, but still not that much. We basically take out
the node as if it's a linked list. i.e.
parent.setRight(parent.getRight().getRight()).

\subsubsection{2 Children}
For example, if we want to remove the node 20 from our tree, we need to find the
largest node on the left side of the tree, which is 18, and remove it and add it
in place of the twenty. This preserves our tree structure.

\section{2015-05-11}
\subsection{Do Now}
Find a way to find the number of nodes in a binary search tree.

It is easy to do this recursively, we simple do the following:

\begin{lstlisting}[language=java]
public int nodeCount() {
    return nodeCountH(root);
}

private int nodeCountH(Node current) {
    if (current == null) {
        return 0;
    }
    else {
        return 1 + nodeCountH(current.getLeft()) + nodeCountH(current.getRight());
    }
}
\end{lstlisting}

Do the following functions:
\begin{enumerate}
    \item Max value from the tree
    \item Height of the tree
    \item Longest leaf-to-leaf path
    \item Split Dupes
    \begin{itemize}
        \item You have a tree, but somewhere in the tree a parent and a child
            have the same value
        \item When you do that you have to insert a node with value of 1 less
            than that value.
    \end{itemize}
\end{enumerate}

\section{2015-05-12}
\subsection{Heap}
A min heap is a binary tree such that all children are greater than their
parents and is as full as possible left and right. A max heap is the same except
all childrens are less than their parents.

\subsubsection{findMin}
Note the finding the min is a min-heap is a constant-time operation. We simply
take the root value.

\subsubsection{removeMin}
However, if we remove the min, we have several problems:
\begin{itemize}
    \item We have a tree without a root
    \item We have a heap that doesn't follow the heap property.
\end{itemize}

To solve this we do the following:
\begin{enumerate}
    \item Remove the root
    \item Replace the root with the lowest rightest value
    \item Pushdown
    \begin{itemize}
        \item swap the parent with the smaller of its children until it reaches
            the leaf.
    \end{itemize}
\end{enumerate}
    
\subsubsection{heapSort}
heapSort can be performed by repeated removing the minimum and using the heap's
self-balancing property to sort the data.

\subsubsection{insert}
Instead of push down, we sift up. We attach the new data to the lowest leftest
node and for each parent we replace it with the smaller of its two children. We
then move up until we hit the root.

\subsection{Algorithm Efficiency}
Find min: $O(1)$

Remove min / pushdown: $O(\log{n})$

Heapsort: $O(n\log{n})$

Insert: $O(\log{n})$
\end{document}
